%
% 6.006 problem set 0 solutions template
%

\documentclass[12pt,twoside]{article}

\input{macros-sp20}
\newcommand{\theproblemsetnum}{0}

\title{6.006 Problem Set 0}

\begin{document}

\handout{Problem Set \theproblemsetnum}{April 28, 2024}

\setlength{\parindent}{0pt}
\medskip\hrulefill\medskip

{\bf Name:} Christopher Jeff Sanchez

\medskip\hrulefill

%%%%%%%%%%%%%%%%%%%%%%%%%%%%%%%%%%%%%%%%%%%%%%%%%%%%%
% See below for common and useful latex constructs. %
%%%%%%%%%%%%%%%%%%%%%%%%%%%%%%%%%%%%%%%%%%%%%%%%%%%%%

% Some useful commands:
% $f(x) = \Theta(x)$
% $T(x, y) \leq \log(x) + 2^y + \binom{2n}{n}$
% \ttt{code\_function}


% You can create unnumbered lists as follows:
% \begin{itemize}
%     \item First item in a list
%         \begin{itemize}
%             \item First item in a list
%                 \begin{itemize}
%                     \item First item in a list
%                     \item Second item in a list
%                 \end{itemize}
%             \item Second item in a list
%         \end{itemize}
%     \item Second item in a list
% \end{itemize}

% You can create numbered lists as follows:
% \begin{enumerate}
%     \item First item in a list
%     \item Second item in a list
%     \item Third item in a list
% \end{enumerate}

% You can write aligned equations as follows:
% \begin{align}
%     \begin{split}
%         (x+y)^3 &= (x+y)^2(x+y) \\
%                 &= (x^2+2xy+y^2)(x+y) \\
%                 &= (x^3+2x^2y+xy^2) + (x^2y+2xy^2+y^3) \\
%                 &= x^3+3x^2y+3xy^2+y^3
%     \end{split}
% \end{align}

% You can create grids/matrices as follows:
% \begin{align}
%     A =
%     \begin{bmatrix}
%         A_{11} & A_{21} \\
%         A_{21} & A_{22}
%     \end{bmatrix}
% \end{align}

% Additional references:
% https://www.cs.put.poznan.pl/ksiek/latexmath.html#set-theory

\begin{problems}

\problem % Problem 1
\begin{align}
    \begin{split}
    A &= \set{ i+\binom{5}{i} \mid i \in Z, \ 0 \leq i \leq 4}  \\
      &= \set{ 0+\binom{5}{0}, 
              \ 1+\binom{5}{1}, 
              \ 2+\binom{5}{2}, 
              \ 3+\binom{5}{3}, 
              \ 4+\binom{5}{4}}  \\
      &= \set{ 1,6,12,13,8 }  \\
    \end{split}
    \\[2ex]
    \begin{split}
    B &= \set{ 3i \mid i \in \{ 1,2,3,4,5\}}  \\
      &= \set{3, 6, 9, 12, 15}  \\
    \end{split}
\end{align}

\begin{problemparts}
    \problempart $A \cap B = \{6, 12\}$ % Problem 1a
    \problempart $|A \cup B| = 7$ % Problem 1b
    \problempart $|A - B| = 3$ % Problem 1c
\end{problemparts}

\problem  % Problem 2
\begin{align}
    \begin{split}
        X &= \{\# \ \text{of heads in three coin flips}\}  \\
          &= \{\text{HHH, HHT, HTH, THH, TTH, THT, HTT, TTT}\}  \\
          &= \{\text{3, 2, 2, 2, 1, 1, 1, 0}\}  \\
    \end{split}
    \\[2ex]
    \begin{split}
        Y &= \{\text{products of two six-sided dice}\}  \\
          &= \{\text{1*1, 1*2, 2*1, ... 6*5, 6*6}\}  \\
          &=
                \begin{tabular}{ c c c c c c}
                 \{1, & 2, & 3, & 4, & 5, & 6,\\ 
               \ \  2, & 4, & 6, & 8, & 10, & 12,\\ 
              \ \   3, & 6, & 9, & 12, & 15, & 18,\\ 
             \ \    4, & 8, & 12, & 16, & 20, & 24,\\ 
           \ \      5, & 10, & 15, & 20, & 25, & 30,\\ 
          \ \       6, & 12, & 18, & 24, & 30, & \ 36\},\\ 
                \end{tabular}
    \end{split}
\end{align}
\begin{problemparts}
    \problempart E$[X] = (3\cdot1 \ + \ 2\cdot3 \ + \ 1\cdot3)/8 = 12/8 = 1.5$% Problem 2a
    \problempart E$[Y] = 426/36 = 11.8\overline{33}$% Problem 2b
    \problempart E$[\{X + Y\}] = 438/44 = 9.9\overline{54}$ % Problem 1a
\end{problemparts}

\problem  % Problem 3
\begin{align}
    \begin{split}
        A = 600/6 = 100
    \end{split}
    \\[2ex]
    \begin{split}
        B = 60\bmod{42} = 17
    \end{split}
\end{align}

\begin{problemparts}
\problempart % Problem 3a
    $A\bmod{2} = 0, \ B\bmod{2} = 0, \ \therefore A \equiv B \pmod{2}$
\problempart % Problem 3b
    $A\bmod{3} = 1, \ B\bmod{3} = 0, \ \therefore A \not\equiv B \pmod{3}$
\problempart % Problem 3c
    $A\bmod{4} = 0, \ B\bmod{4} = 2, \ \therefore A \not\equiv B \pmod{4}$
\end{problemparts}

% Problem 4
\problem Prove \textbf{by induction}  that 
         $\sum_{i=1}^{n} i^3 = [\frac{n(n+1)}{2}]^2  $,
         for any $n \geq 1$. \\ 
\\
\proof Let $P(n): \sum_{i=1}^{n} i^3 = [\frac{n(n+1)}{2}]^2  $.\\
    \\
    Base case, $P(1)$:  
    \begin{align}
        \sum_{i=1}^{1} i^3 = 1 \\
        \biggl[\frac{1(1+1)}{2}\biggr]^2 = 1
    \end{align}
    Hence, base case is true. For the induction step, assuming $P(n)$
    is true, we get $P(n+1)$:
    \begin{align}
        \sum_{i=1}^{n+1} i^{3} &= (n+1)^{3} + \sum_{i=1}^{n} i^{3}  \\
                                &= (n+1)^{3} + \biggl[\frac{n(n+1)}{2}\biggr]^2  \\
                                &= \frac{4(n+1)^3 + n^2(n+1)^2}{4}
                                 = \frac{(4(n+1) + n^2)(n+1)^2}{4} \\
                                &= \frac{(n^2 + 4n + 1)(n+1)^2}{4}  
                                 = \frac{(n+2)^2(n+1)^2}{4} \\
                                &= \biggl[\frac{(n+1)((n+1) + 1)}{2}\biggr]^{2}
    \end{align}

\null\hfill $\blacksquare$



\newpage
% Problem 5
\problem Prove \textbf{by induction}  that every connected undirected graph $G=(V,E)$
for which $|E|=|V|-1$ is acyclic. \\
\\
\textit{Recall:}
    \begin{itemize}
        \item \ A graph is connected if each pair of vertices has at least one
            connecting edge.
        \item \ An undirected graph simply means that all edges has no direction
            component.
        \item \ An acyclic graph is a graph that contains no cycles. When
            traversing the graph from vertex to vertex, no same vertex shall be
            visited twice.
        \item \ $|E|=|V|-1$ means that there is one less edge as there are
            vertices.
    \end{itemize}

\textit{Proof.} Using induction on the number of vertices, $n=|V|$, in a graph
  $G=(V,E)$. Define $P(n)$ as: "Given that G is a connected undirected graph, 
  if $|E|=|V|-1$, then $G$ is acyclic." \\
  \\
  Base case, $P(1)$: $|V|=1$, hence $|E|=0$. A graph with only one vertex can't
  create a cycle since a cycle requires a nonempty seqyence of edges. Thus the
  base case holds.\\
  \\
  For the induction step, we consider a "shrink-down, build-up" approach: 
      \footnote{This is to avoid the "Build-up Error," where one assumes
      that \underline{every size n+1 graph with some property} can be built-up 
      from \underline{a size n graph with the same property.} This assumption is 
      not true by default. Note that the property in question is that of 
      the antecedent, which is necessary to assume $P(n)$.}
  Consider an (n+1) vertex graph $G'$ that is connected, undirected, and has
  $|V|=n+1$, and $|E|=n$. We know that any connected graph has a spanning tree,
  and a leaf of that spanning tree can be removed such that we are left with a
  graph $G$ that is still connected, undirected, and has $|V|=n$, and $|E|=n-1$.
  If we assume $P(n)$, then $G$ is acyclic. Now we see if we can get $G'$ from
  $G$ by adding a vertex.\\
  \\
  \textbf{Case 1:} Connecting the vertex $v_{n+1}$ to an edge in $G$. This is
  not possible since all edges in G are already connected to 2 vertices (since
  it is a connected graph with one less edge than vertices).\\
  \\
  \textbf{Case 2:} Connecting the edge $e_n$ to two vertices already in G. This
  is not possible since if $e_n=\{v_{n-1}, v_n\}$, then by case 1, there won't
  be any edge for $v_{n+1}$ and it won't be a part of the connected graph G.\\
  \\
  \textbf{Case 3:} Connecting the new vertex and edge together with a vertex in
  G. That is, $e_n = \{v_n, v_{n+1}\}$. Since G is said to be acyclic, we know
  that $v_n$ doesn't belong to any prior cycle. For $v_n$ and $v_{n+1}$ to
  create a new cycle, there should be a path from $v_n$ to $v_{n+1}$ and back
  to $v_n$. Since $v_{n+1}$ is only connected to $e_{n}$, then there is no path
  from $v_{n+1}$ back to $v_{n}$ without repeating the edge. Therefore, the two
  vertices don't create a new cycle.\\
  \\
  By considering the three cases, we see that $G$ remains ayclic.\\
  \null\hfill $\blacksquare$



  



\vfill
\problem  % Problem 6
Submit your implementation to {\small\url{alg.mit.edu}}.

\begin{lstlisting}
def count_long_subarray(A):
    '''
    Input:  A     | Python Tuple of positive integers
    Output: count | number of longest increasing subarrays of A
    '''
    count = 0
    prev_a = 0
    current_subarray_size = 0
    max_subarray_size = 0

    for a in A:
        print("a =", a)

        if a > prev_a:  # while increasing
            # keep track of length of increasing-subarray
            current_subarray_size += 1
            print("subarray_size =", current_subarray_size)
            prev_a = a

            if current_subarray_size > max_subarray_size:
                max_subarray_size = current_subarray_size
                count = 1
                print(
                    "update max_subarray_size =",
                    max_subarray_size,
                    "reset count =",
                    count,
                )
            elif current_subarray_size == max_subarray_size:
                count += 1
                print("increment count =", count)

            continue

        # if not inreasing anymore, end of subarray
        prev_a = a  # the update in the 1st if-case won't be executed, update here
        current_subarray_size = 1  # 1 because current_subarray = [ prev_a ]
        print("reset current_subarray_size =", 1)

    print("COUNT =", count)
    return count
\end{lstlisting}

\end{problems}

\end{document}
