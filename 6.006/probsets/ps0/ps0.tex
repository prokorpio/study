%
% 6.006 problem set 0 solutions template
%

\documentclass[12pt,twoside]{article}

\input{macros-sp20}
\newcommand{\theproblemsetnum}{0}

\title{6.006 Problem Set 0}

\begin{document}

\handout{Problem Set \theproblemsetnum}{April 28, 2024}

\setlength{\parindent}{0pt}
\medskip\hrulefill\medskip

{\bf Name:} Christopher Jeff Sanchez

\medskip\hrulefill

%%%%%%%%%%%%%%%%%%%%%%%%%%%%%%%%%%%%%%%%%%%%%%%%%%%%%
% See below for common and useful latex constructs. %
%%%%%%%%%%%%%%%%%%%%%%%%%%%%%%%%%%%%%%%%%%%%%%%%%%%%%

% Some useful commands:
% $f(x) = \Theta(x)$
% $T(x, y) \leq \log(x) + 2^y + \binom{2n}{n}$
% \ttt{code\_function}


% You can create unnumbered lists as follows:
% \begin{itemize}
%     \item First item in a list
%         \begin{itemize}
%             \item First item in a list
%                 \begin{itemize}
%                     \item First item in a list
%                     \item Second item in a list
%                 \end{itemize}
%             \item Second item in a list
%         \end{itemize}
%     \item Second item in a list
% \end{itemize}

% You can create numbered lists as follows:
% \begin{enumerate}
%     \item First item in a list
%     \item Second item in a list
%     \item Third item in a list
% \end{enumerate}

% You can write aligned equations as follows:
% \begin{align}
%     \begin{split}
%         (x+y)^3 &= (x+y)^2(x+y) \\
%                 &= (x^2+2xy+y^2)(x+y) \\
%                 &= (x^3+2x^2y+xy^2) + (x^2y+2xy^2+y^3) \\
%                 &= x^3+3x^2y+3xy^2+y^3
%     \end{split}
% \end{align}

% You can create grids/matrices as follows:
% \begin{align}
%     A =
%     \begin{bmatrix}
%         A_{11} & A_{21} \\
%         A_{21} & A_{22}
%     \end{bmatrix}
% \end{align}

% Additional references:
% https://www.cs.put.poznan.pl/ksiek/latexmath.html#set-theory

\begin{problems}

\problem % Problem 1
\begin{align}
    \begin{split}
    A &= \set{ i+\binom{5}{i} \mid i \in Z, \ 0 \leq i \leq 4}  \\
      &= \set{ 0+\binom{5}{0}, 
              \ 1+\binom{5}{1}, 
              \ 2+\binom{5}{2}, 
              \ 3+\binom{5}{3}, 
              \ 4+\binom{5}{4}}  \\
      &= \set{ 1,6,12,13,8 }  \\
    \end{split}
    \\[2ex]
    \begin{split}
    B &= \set{ 3i \mid i \in \{ 1,2,3,4,5\}}  \\
      &= \set{3, 6, 9, 12, 15}  \\
    \end{split}
\end{align}

\begin{problemparts}
    \problempart $A \cap B = \{6, 12\}$ % Problem 1a
    \problempart $|A \cup B| = 7$ % Problem 1b
    \problempart $|A - B| = 3$ % Problem 1c
\end{problemparts}

\problem  % Problem 2
\begin{align}
    \begin{split}
        X &= \{\# \ \text{of heads in three coin flips}\}  \\
          &= \{\text{HHH, HHT, HTH, THH, TTH, THT, HTT, TTT}\}  \\
          &= \{\text{3, 2, 2, 2, 1, 1, 1, 0}\}  \\
    \end{split}
    \\[2ex]
    \begin{split}
        Y &= \{\text{products of two six-sided dice}\}  \\
          &= \{\text{1*1, 1*2, 2*1, ... 6*5, 6*6}\}  \\
          &=
                \begin{tabular}{ c c c c c c}
                 \{1, & 2, & 3, & 4, & 5, & 6,\\ 
               \ \  2, & 4, & 6, & 8, & 10, & 12,\\ 
              \ \   3, & 6, & 9, & 12, & 15, & 18,\\ 
             \ \    4, & 8, & 12, & 16, & 20, & 24,\\ 
           \ \      5, & 10, & 15, & 20, & 25, & 30,\\ 
          \ \       6, & 12, & 18, & 24, & 30, & \ 36\},\\ 
                \end{tabular}
    \end{split}
\end{align}
\begin{problemparts}
    \problempart E$[X] = (3\cdot1 \ + \ 2\cdot3 \ + \ 1\cdot3)/8 = 12/8 = 1.5$% Problem 2a
    \problempart E$[Y] = 426/36 = 11.8\overline{33}$% Problem 2b
    \problempart E$[\{X + Y\}] = 438/44 = 9.9\overline{54}$ % Problem 1a
\end{problemparts}

\problem  % Problem 3
\begin{align}
    \begin{split}
        A = 600/6 = 100
    \end{split}
    \\[2ex]
    \begin{split}
        B = 60\bmod{42} = 17
    \end{split}
\end{align}

\begin{problemparts}
\problempart % Problem 3a
    $A\bmod{2} = 0, \ B\bmod{2} = 0, \ \therefore A \equiv B \pmod{2}$
\problempart % Problem 3b
    $A\bmod{3} = 1, \ B\bmod{3} = 0, \ \therefore A \not\equiv B \pmod{3}$
\problempart % Problem 3c
    $A\bmod{4} = 0, \ B\bmod{4} = 2, \ \therefore A \not\equiv B \pmod{4}$
\end{problemparts}

% Problem 4
\problem Prove \textbf{by induction}  that 
         $\sum_{i=1}^{n} i^3 = [\frac{n(n+1)}{2}]^2  $,
         for any $n \geq 1$. \\ 
\\
\proof Let $P(n): \sum_{i=1}^{n} i^3 = [\frac{n(n+1)}{2}]^2  $.\\
    \\
    Base case, $P(1)$:  
    \begin{align}
        \sum_{i=1}^{1} i^3 = 1 \\
        \biggl[\frac{1(1+1)}{2}\biggr]^2 = 1
    \end{align}
    Hence, base case is true. For the induction step, assuming $P(n)$
    is true, we get $P(n+1)$:
    \begin{align}
        \sum_{i=1}^{n+1} i^{3} &= (n+1)^{3} + \sum_{i=1}^{n} i^{3}  \\
                                &= (n+1)^{3} + \biggl[\frac{n(n+1)}{2}\biggr]^2  \\
                                &= \frac{4(n+1)^3 + n^2(n+1)^2}{4}
                                 = \frac{(4(n+1) + n^2)(n+1)^2}{4} \\
                                &= \frac{(n^2 + 4n + 1)(n+1)^2}{4}  
                                 = \frac{(n+2)^2(n+1)^2}{4} \\
                                &= \biggl[\frac{(n+1)((n+1) + 1)}{2}\biggr]^{2}
    \end{align}

$\hfill \blacksquare$



\newpage
\problem  % Problem 5

\vfill
\problem  % Problem 6
Submit your implementation to {\small\url{alg.mit.edu}}.

\begin{lstlisting}
def count_long_subarray(A):
    '''
    Input:  A     | Python Tuple of positive integers
    Output: count | number of longest increasing subarrays of A
    '''
    count = 0
    ##################
    # YOUR CODE HERE #
    ##################
    return count
\end{lstlisting}

\end{problems}

\end{document}
